% ------------------------------------------------------------------------------
% Michael Kastl 87 28 60
% ------------------------------------------------------------------------------
%
%
% Dokumentenkopf 
% ------------------------------------------------------------------------------
\documentclass[
    11pt, % Schriftgr��e
    DIV10,
    ngerman, % f�r Umlaute, Silbentrennung etc.
    a4paper, % Papierformat
    oneside, % einseitiges Dokument
    titlepage, % es wird eine Titelseite verwendet
    parskip=half, % Abstand zwischen Abs�tzen (halbe Zeile)
    headings=normal, % Gr��e der �berschriften verkleinern
    listof=totoc, % Verzeichnisse im Inhaltsverzeichnis auff�hren
    bibliography=totoc, % Literaturverzeichnis im Inhaltsverzeichnis auff�hren
    index=totoc, % Index im Inhaltsverzeichnis auff�hren
    captions=tableheading, % Beschriftung von Tabellen unterhalb ausgeben
    final, % Status des Dokuments (final/draft)
    numbers=noenddot
]{scrreprt}

\renewcommand*{\chapterheadstartvskip}{\vspace{-1\baselineskip}} %Abstand vor Section
\setlength{\headsep}{30pt} %Abstand Kopfzeile Textk�rper


%Abstand vor und nach �berschrift einstellbar
%\makeatletter
%\def\@makeschapterhead#1{%
% \vspace*{5\p@}%       % 5 = Abstand vor chapter
%  {\parindent \z@ \raggedright
%    \normalfont
%    \interlinepenalty\@M
%    \huge \bfseries  #1\par\nobreak 
%    \vskip 30\p@		% 30 = Abstand nach chapter
%  }}
%\makeatother


% Meta-Informationen -----------------------------------------------------------
%   Informationen �ber das Dokument, wie z.B. Titel, Autor, Matrikelnr. etc
%   werden in der Datei Meta.tex definiert und k\"onnen danach global
%   verwendet werden.
% ------------------------------------------------------------------------------
% Meta-Informationen -----------------------------------------------------------
%   Definition von globalen Parametern, die im gesamten Dokument verwendet
%   werden k�nnen (z.B auf dem Deckblatt etc.).
%
%   ACHTUNG: Wenn die Texte Umlaute oder ein Esszet enthalten, muss der folgende
%            Befehl bereits an dieser Stelle aktiviert werden:
%            \usepackage[latin1]{inputenc}
% ------------------------------------------------------------------------------
\newcommand{\titel}{Vorlage f\"ur eine Diplomarbeit in LaTeX}
\newcommand{\untertitel}{\ }
\newcommand{\art}{Diplomarbeit}
\newcommand{\fachgebiet}{Automatisierungstechnik}
\newcommand{\autor}{Max Mustermann}
\newcommand{\studienbereich}{Mechatronik}
\newcommand{\matrikelnr}{xx xx xx}
\newcommand{\erstgutachter}{Prof. Dr.-Ing. Max Mustermann}
\newcommand{\zweitgutachter}{Dipl.-Ing. xxx}
\newcommand{\jahr}{2010}
\newcommand{\ort}{M"unchen}
\newcommand{\logo}{mkDoc-Logo.png}


% ben\"otigte Packages -----------------------------------------------------------
%   LaTeX-Packages, die ben�tigt werden, sind in die Datei Packages.tex
%   "ausgelagert", um diese Vorlage m�glichst �bersichtlich zu halten.
% ------------------------------------------------------------------------------
\input{Packages}

% Erstellung eines Index und Abk�rzungsverzeichnisses aktivieren ---------------

\makeindex
%\makenomenclature

% Kopf- und Fu�zeilen, Seitenr�nder etc. ---------------------------------------
\input{Seitenstil}

% eigene Definitionen f�r Silbentrennung
\include{Silbentrennung}

% eigene LaTeX-Befehle
\include{Befehle}

% Das eigentliche Dokument -----------------------------------------------------
%   Der eigentliche Inhalt des Dokuments beginnt hier. Die einzelnen Seiten
%   und Kapitel werden in eigene Dateien ausgelagert und hier nur inkludiert.
% ------------------------------------------------------------------------------
\begin{document}

% auch subsubsection nummerieren
\setcounter{secnumdepth}{3}
\setcounter{tocdepth}{3}

% Deckblatt und Abstract ohne Seitenzahl
\ofoot{}
\include{Deckblatt}
\include{Inhalt/Abstract}
\ofoot{\pagemark}

% Seitennummerierung -----------------------------------------------------------
%   Vor dem Hauptteil werden die Seiten in gro�en r�mischen Ziffern 
%   nummeriert.
% ------------------------------------------------------------------------------
\pagenumbering{Roman}
\tableofcontents % Inhaltsverzeichnis

% Abk�rzungsverzeichnis --------------------------------------------------------

%\chapter*
\addchap{Abk\"urzungsverzeichnis}
\begin{acronym}[laaaaaaaaaaang] %In der Eckigen Klammer muss das l�ngste Akronym stehen. Es wird dann die gesamte weitere Tabelle danach ausgerichtet
  \acro{AM}{andere Ma\ss nahmen}
  \acro{B10d}[$\text{B}_{\text{10d}}$]{Anzahl von Zyklen, bis 10 \% der Komponenten gef\"ahrlich ausgefallen sind (f\"ur pneumatische und elektromechanische Komponenten)}
  \acro{CCF}{Ausfall aufgrund gemeinsamer Ursache}
  \acro{DC}{Diagnosedeckungsgrad}
  \acro{DCavg}[$\text{DC}_{\text{avg}}$]{durchschnittlicher Diagnosedeckungsgrad}
  \acro{dop}[$\text{d}_{\text{op}}$]{mittlere Betriebszeit in Tagen je Jahr}
  \acro{EMC}{Elektromagnetische Vertr\"aglichkeit}
  \acro{FMEA}{Fehlzustandsart- und -auswirkungsanalyse}
  \acro{hop}[$\text{h}_{\text{op}}$]{mittlere Betriebszeit in Stunden je Tag}
  \acro{MTTF}{mittlere Zeit bis zum Ausfall}
  \acro{MTTFd}[$\text{MTTF}_{\text{d}}$]{mittlere Zeit bis zum gefahrbringenden Ausfall}
  \acro{MTTFdC1}[$\text{MTTF}_{\text{dC1}}$]{mittlere Zeit bis zum gefahrbringenden Ausfall Kanal 1}
  \acro{MTTFdC2}[$\text{MTTF}_{\text{dC2}}$]{mittlere Zeit bis zum gefahrbringenden Ausfall Kanal 2}
  \acro{MTTFdi}[$\text{MTTF}_{\text{di}}$]{mittlere Zeit bis zum gefahrbringenden Ausfall eines Bauteils}
  \acro{nop}[$\text{n}_{\text{op}}$]{mittlere Anzahl j\"ahrlicher Bet\"atigungen}
  \acro{PFHD}[$\text{PFH}_{\text{d}}$]{Wahrscheinlichkeit eines gefahrbringenden Ausfalls pro Stunde}
  \acro{PL}{Performance Level}
  \acro{PLr}[$\text{PL}_{\text{r}}$]{erforderliches Performance Level}
  \acro{SIL}{Safety integrated Level}
  \acro{SLS}{sicher begrenzte Geschwindigkeit}
  \acro{SOS}{Stillstands\"uberwachung}
  \acro{SRCF}{sicherheitsbezogene Steuerungsfunktion}
  \acro{SRECS}{sicherheitsbezogenes elektrisches Steuerungssystem}
  \acro{SRP/CS}{sicherheitsbezogenes Teil einer Steuerung}
  \acro{STO}{sicherer Halt} 
  \acro{tzyklus}[$\text{t}_{\text{Zyklus}}$]{mittlere Zeit zwischen dem Beginn zweier aufeinander folgenden Zyklen des Bauteils (z. B. Schalten eines Ventils) in Sekunden je Zyklus}
  \acro{VDMA}{Verband Deutscher Maschinen- und Anlagenbau}
  \acro{lambdad}[$\lambda_{\text{d}}$]{Anzahl aller gef\"ahrlichen Ausf\"alle}
  \acro{lambdadd}[$\lambda_{\text{dd}}$]{Anzahl aller erkannter gef\"ahrlichen Ausf\"alle}
\end{acronym}

\listoffigures % Abbildungsverzeichnis
\listoftables % Tabellenverzeichnis
%\renewcommand{\lstlistlistingname}{Verzeichnis der Listings}
%\lstlistoflistings % Listings-Verzeichnis

% arabische Seitenzahlen im Hauptteil ------------------------------------------
\clearpage
\pagenumbering{arabic}

% die Inhaltskapitel werden in "Inhalt.tex" inkludiert -------------------------
% Hier k�nnen die einzelnen Kapitel inkludiert werden. Sie m�ssen in den 
% entsprechenden .TEX-Dateien vorliegen. Die Dateinamen k�nnen nat�rlich 
% angepasst werden.
%%\chapter*
\addchap{Abk\"urzungsverzeichnis}
\begin{acronym}[laaaaaaaaaaang] %In der Eckigen Klammer muss das l�ngste Akronym stehen. Es wird dann die gesamte weitere Tabelle danach ausgerichtet
  \acro{AM}{andere Ma\ss nahmen}
  \acro{B10d}[$\text{B}_{\text{10d}}$]{Anzahl von Zyklen, bis 10 \% der Komponenten gef\"ahrlich ausgefallen sind (f\"ur pneumatische und elektromechanische Komponenten)}
  \acro{CCF}{Ausfall aufgrund gemeinsamer Ursache}
  \acro{DC}{Diagnosedeckungsgrad}
  \acro{DCavg}[$\text{DC}_{\text{avg}}$]{durchschnittlicher Diagnosedeckungsgrad}
  \acro{dop}[$\text{d}_{\text{op}}$]{mittlere Betriebszeit in Tagen je Jahr}
  \acro{EMC}{Elektromagnetische Vertr\"aglichkeit}
  \acro{FMEA}{Fehlzustandsart- und -auswirkungsanalyse}
  \acro{hop}[$\text{h}_{\text{op}}$]{mittlere Betriebszeit in Stunden je Tag}
  \acro{MTTF}{mittlere Zeit bis zum Ausfall}
  \acro{MTTFd}[$\text{MTTF}_{\text{d}}$]{mittlere Zeit bis zum gefahrbringenden Ausfall}
  \acro{MTTFdC1}[$\text{MTTF}_{\text{dC1}}$]{mittlere Zeit bis zum gefahrbringenden Ausfall Kanal 1}
  \acro{MTTFdC2}[$\text{MTTF}_{\text{dC2}}$]{mittlere Zeit bis zum gefahrbringenden Ausfall Kanal 2}
  \acro{MTTFdi}[$\text{MTTF}_{\text{di}}$]{mittlere Zeit bis zum gefahrbringenden Ausfall eines Bauteils}
  \acro{nop}[$\text{n}_{\text{op}}$]{mittlere Anzahl j\"ahrlicher Bet\"atigungen}
  \acro{PFHD}[$\text{PFH}_{\text{d}}$]{Wahrscheinlichkeit eines gefahrbringenden Ausfalls pro Stunde}
  \acro{PL}{Performance Level}
  \acro{PLr}[$\text{PL}_{\text{r}}$]{erforderliches Performance Level}
  \acro{SIL}{Safety integrated Level}
  \acro{SLS}{sicher begrenzte Geschwindigkeit}
  \acro{SOS}{Stillstands\"uberwachung}
  \acro{SRCF}{sicherheitsbezogene Steuerungsfunktion}
  \acro{SRECS}{sicherheitsbezogenes elektrisches Steuerungssystem}
  \acro{SRP/CS}{sicherheitsbezogenes Teil einer Steuerung}
  \acro{STO}{sicherer Halt} 
  \acro{tzyklus}[$\text{t}_{\text{Zyklus}}$]{mittlere Zeit zwischen dem Beginn zweier aufeinander folgenden Zyklen des Bauteils (z. B. Schalten eines Ventils) in Sekunden je Zyklus}
  \acro{VDMA}{Verband Deutscher Maschinen- und Anlagenbau}
  \acro{lambdad}[$\lambda_{\text{d}}$]{Anzahl aller gef\"ahrlichen Ausf\"alle}
  \acro{lambdadd}[$\lambda_{\text{dd}}$]{Anzahl aller erkannter gef\"ahrlichen Ausf\"alle}
\end{acronym}

\include{Inhalt/Einleitung}
\include{Inhalt/Vorgaben}
\include{Inhalt/Entwicklung}
\include{Inhalt/Analyse}
\chapter{Chapter 5}
\label{cha:Chapter5}

\section{Section 1}
\label{sec:5Section1}
Willkommen im Portal f"ur Elektronik, Maschinenbau und Mechatronik !
Dieses Portal soll euch beim lernen und diskutieren der einzelnen Studienf"acher behilflich sein, oder im Alltag als Knowledge-Base zur Verf"ugung stehen ! \\
F"ur jedes Studienfach wird in einem "Ubersichtsartikel der Inhalt zusammengefasst und die einzelnen Fachartikel in Beziehung zueinader gestellt. Kommen Formeln in den Fachartikeln vor, werden diese in einer Formelsammlung zu dem jeweiligen Studienfach zusammengef"uhrt. Am Ende soll jedes Studienfach einen "Ubersichtsartikel und wenn m"oglich eine Formelsammlung besitzen.

\subsection{Subcestion 1.1}
\label{subsec:5Subcestion1.1}

\begin{figure}[htb]
\centering
\includegraphics[width=0.2\textwidth]{mkDoc-Logo.png}
\caption{mkDoc}
\label{fig:mkDoc}
\end{figure}


\subsection{Subcestion 1.2}
\label{subsec:5Subcestion 1.2}
Willkommen im Portal f"ur Elektronik, Maschinenbau und Mechatronik !\footnote{\Vgl\Zitat[S.~11]{Sicherheitstechnik}}
Dieses Portal soll euch beim lernen und diskutieren der einzelnen Studienf"acher behilflich sein, oder im Alltag als Knowledge-Base zur Verf"ugung stehen ! \\
F"ur jedes Studienfach wird in einem "Ubersichtsartikel der Inhalt zusammengefasst und die einzelnen Fachartikel in Beziehung zueinader gestellt. Kommen Formeln in den Fachartikeln vor, werden diese in einer Formelsammlung zu dem jeweiligen Studienfach zusammengef"uhrt. Am Ende soll jedes Studienfach einen "Ubersichtsartikel und wenn m"oglich eine Formelsammlung besitzen.

\section{Section 2}
\label{sec:5Section2}

Willkommen im Portal f"ur Elektronik, Maschinenbau und Mechatronik !
Dieses Portal soll euch beim lernen und diskutieren der einzelnen Studienf"acher behilflich sein, oder im Alltag als Knowledge-Base zur Verf"ugung stehen ! \ref{sec:5Section1} \\
F"ur jedes Studienfach wird in einem "Ubersichtsartikel der Inhalt zusammengefasst und die einzelnen Fachartikel in Beziehung zueinader gestellt. Kommen Formeln in den Fachartikeln vor, werden diese in einer Formelsammlung zu dem jeweiligen Studienfach zusammengef"uhrt. Am Ende soll jedes Studienfach einen "Ubersichtsartikel und wenn m"oglich eine Formelsammlung besitzen.

\section{Section 3}
\label{sec:5Section3}

\subsection{Subcestion 3.1}
\label{subsec:4Subcestion3.1}
Willkommen im Portal f"ur Elektronik, Maschinenbau und Mechatronik !
Dieses Portal soll euch beim lernen und diskutieren der einzelnen Studienf"acher behilflich sein, oder im Alltag als Knowledge-Base zur Verf"ugung stehen ! \\
F"ur jedes Studienfach wird in einem "Ubersichtsartikel der Inhalt zusammengefasst und die einzelnen Fachartikel in Beziehung zueinader gestellt. Kommen Formeln in den Fachartikeln vor, werden diese in einer Formelsammlung zu dem jeweiligen Studienfach zusammengef"uhrt. Am Ende soll jedes Studienfach einen "Ubersichtsartikel und wenn m"oglich eine Formelsammlung besitzen. \acs{MTTF}

\subsection{Subcestion 3.2}
\label{subsec:5Subcestion3.2}
Willkommen im Portal f"ur Elektronik, Maschinenbau und Mechatronik !
Dieses Portal soll euch beim lernen und diskutieren der einzelnen Studienf"acher behilflich sein, oder im Alltag als Knowledge-Base zur Verf"ugung stehen ! \\
F"ur jedes Studienfach wird in einem "Ubersichtsartikel der Inhalt zusammengefasst und die einzelnen Fachartikel in Beziehung zueinader gestellt. Kommen Formeln in den Fachartikeln vor, werden diese in einer Formelsammlung zu dem jeweiligen Studienfach zusammengef"uhrt. Am Ende soll jedes Studienfach einen "Ubersichtsartikel und wenn m"oglich eine Formelsammlung besitzen.

\newpage
\subsection{Subcestion 3.3}
\label{subsec:5Subcestion3.3}

\begin{longtable}{|p{9cm}|c|}
\caption{Klassifikation der Schwere (S)}\footnote{\Vgl\Zitat[S.~76]{DINEN62061}}\\
\hline
\label{tab:KlassifikationSchwere}
\textbf{Auswirkungen} & \textbf{Schwere (S)}\\
\hline
\hline
irreversibel: Tod, Verlust eines Auges oder Arms & 4\\
\hline
irreversibel: gebrochene Gliedma\ss en, Verlust (eines) mehrerer Finger(s) & 3\\
\hline
reversibel: Behandlung durch einen Mediziner erforderlich & 2\\
\hline
reversibel: Erste Hilfe erforderlich & 1\\
\hline
\end{longtable}

%\include{Inhalt/ZitateReferenzen}
%\chapter{Bilder und Listings}

\section{pasCAL}
\label{sec:pasCAL}

\section{SISTEMA}
\label{sec:SISTEMA}


\include{Inhalt/Fazit}
\chapter{Zusammenfassung}
\label{cha:Zusammenfassung}

\section{Section 1}
\label{sec:7Section1}
Willkommen im Portal f"ur Elektronik, Maschinenbau und Mechatronik !
Dieses Portal soll euch beim lernen und diskutieren der einzelnen Studienf"acher behilflich sein, oder im Alltag als Knowledge-Base zur Verf"ugung stehen ! \\
F"ur jedes Studienfach wird in einem "Ubersichtsartikel der Inhalt zusammengefasst und die einzelnen Fachartikel in Beziehung zueinader gestellt. Kommen Formeln in den Fachartikeln vor, werden diese in einer Formelsammlung zu dem jeweiligen Studienfach zusammengef"uhrt. Am Ende soll jedes Studienfach einen "Ubersichtsartikel und wenn m"oglich eine Formelsammlung besitzen.

\subsection{Subcestion 1.1}
\label{subsec:7Subcestion1.1}

\begin{figure}[htb]
\centering
\includegraphics[width=0.2\textwidth]{mkDoc-Logo.png}
\caption{mkDoc}
\label{fig:mkDoc}
\end{figure}


\subsection{Subcestion 1.2}
\label{subsec:7Subcestion 1.2}
Willkommen im Portal f"ur Elektronik, Maschinenbau und Mechatronik !\footnote{\Vgl\Zitat[S.~11]{Sicherheitstechnik}}
Dieses Portal soll euch beim lernen und diskutieren der einzelnen Studienf"acher behilflich sein, oder im Alltag als Knowledge-Base zur Verf"ugung stehen ! \\
F"ur jedes Studienfach wird in einem "Ubersichtsartikel der Inhalt zusammengefasst und die einzelnen Fachartikel in Beziehung zueinader gestellt. Kommen Formeln in den Fachartikeln vor, werden diese in einer Formelsammlung zu dem jeweiligen Studienfach zusammengef"uhrt. Am Ende soll jedes Studienfach einen "Ubersichtsartikel und wenn m"oglich eine Formelsammlung besitzen.

\chapter{Vom Autor verwendete Software}
\label{cha:Werkzeuge}

\begin{itemize}
\itemd{Inkscape}{Version: 0.47,
\newline Website: \url{http://www.inkscape.org/download/?lang=de},
\newline Stand: 09.08.2009.}
\itemd{LTi SafePLC}{Version: 1.25,
\newline Website: \url{http://www.lust-antriebstechnik.de/},
\newline Stand: 09.11.2009.}
\itemd{Pilz PAScal}{Version: v1.3.2 Build 004,
\newline Website: \url{http://www.pilz.com/downloads/restricted/pascal_1.3.2.zip},
\newline Stand: 14.10.2009.}
\itemd{SISTEMA}{Version: 1.0.5,\newline Website: \url{http://www.dguv.de/bgia/de/pra/softwa/sistema/index.jsp},
\newline Stand: 14.10.2009.}
\itemd{TeXShop \LaTeX }{Version: 2.26 Release 3/17/2009,
\newline Website: \url{http://www.uoregon.edu/~koch/texshop/index.html},
\newline Stand: 12.09.2009.}
\end{itemize}



% Literaturverzeichnis ---------------------------------------------------------
%   Das Literaturverzeichnis wird aus der BibTeX-Datenbank "Bibliographie.bib"
%   erstellt.
% ------------------------------------------------------------------------------
\bibliography{Bibliographie} % Aufruf: bibtex Diplomarbeit
\bibliographystyle{natdin} % DIN-Stil des Literaturverzeichnisses

% Index ------------------------------------------------------------------------
%   Zum Erstellen eines Index, die folgende Zeile auskommentieren.
% ------------------------------------------------------------------------------
%\printindex

\addchap{Eidesstattliche Erkl"arung}
Ich, \autor, Matrikel-Nr.\ \matrikelnr, versichere hiermit, dass ich meine Diplomarbeit mit dem Thema
\begin{quote}
\textit{\titel} \textit{\untertitel}
\end{quote}
selbstst"andig verfasst und keine anderen als die angegebenen Quellen und Hilfsmittel benutzt habe, wobei ich alle w�rtlichen und sinngem"a\ss en Zitate als solche gekennzeichnet habe. Die Arbeit wurde bisher keiner anderen Pr"ufungsbeh"orde vorgelegt und auch nicht ver"offentlicht.

Mir ist bekannt, dass ich meine Diplomarbeit zusammen mit dieser Erkl"arung fristgem"a\ss \ nach Vergabe des Themas in dreifacher Ausfertigung und gebunden im Pr"ufungsamt der Wilhelm-B\"uchner-Hochschule abzugeben oder sp"atestens mit dem Poststempel des Tages, an dem die Frist abl"auft, zu senden habe.\\[6ex]

\ort, den 
\newline

\rule[-0.2cm]{5cm}{0.5pt}

\textsc{\autor} 
 % Selbst�ndigkeitserkl�rung

% Anhang -----------------------------------------------------------------------
%   Die Inhalte des Anhangs werden analog zu den Kapiteln inkludiert.
%   Dies geschieht in der Datei "Anhang.tex".
% ------------------------------------------------------------------------------
\begin{appendix}
    \clearpage
    \pagenumbering{roman}
    \chapter{Anhang}
    \label{sec:Anhang}
    % Rand der Aufz�hlungen in Tabellen anpassen
    \setdefaultleftmargin{1em}{}{}{}{}{}
    \section{DVD}
\label{sec:DVD}

\begin{description}
\item[Diplomarbeit]\begin{it}PDF-Datei und Anh"ange \end{it}
\begin{itemize}
\item{Diplomarbeit} 
\item{Dokument1}
\item{Dokument2} 
\item{Dokument3}
\item{Dokument4}
\item{Dokument5}
\item{Dokument6}
\item{Dokument7}
\end{itemize}
\item[Sonstiges]\begin{it}Zus"atzliche Informationonen zur Diplomarbeit \end{it}
\begin{itemize}
\item{Dokument1}
\item{Dokument2} 
\item{Dokument3}
\end{itemize}
\end{description}



\section{mkDoc2}
\label{sec:mkDoc2}
\vspace{4mm}
\includegraphics[width=0.3\textwidth]{mkDoc-Logo.png}

\section{mkDoc3}
\label{sec:mkDoc3}
\vspace{4mm}
\includegraphics[width=0.3\textwidth]{mkDoc-Logo.png}


\end{appendix}


\end{document}
