\documentclass[fontsize=12pt, paper=a4, headinclude, twoside=false, parskip=half+, pagesize=auto, numbers=noenddot, plainheadsepline, open=right, toc=listof, toc=bibliography]{scrreprt}
% PDF-Kompression
\pdfminorversion=5
\pdfobjcompresslevel=1
% Allgemeines
\usepackage[automark]{scrpage2} % Kopf- und Fußzeilen
\usepackage{amsmath,marvosym} % Mathesachen
\usepackage[T1]{fontenc} % Ligaturen, richtige Umlaute im PDF
\usepackage[utf8]{inputenc}% UTF8-Kodierung für Umlaute usw
% Schriften
\usepackage{mathpazo} % Palatino für Mathemodus
%\usepackage{mathpazo,tgpagella} % auch sehr schöne Schriften
\usepackage{setspace} % Zeilenabstand
\onehalfspacing % 1,5 Zeilen
% Schriften-Größen
\setkomafont{chapter}{\Huge\rmfamily} % Überschrift der Ebene
\setkomafont{section}{\Large\rmfamily}
\setkomafont{subsection}{\large\rmfamily}
\setkomafont{subsubsection}{\large\rmfamily}
\setkomafont{chapterentry}{\large\rmfamily} % Überschrift der Ebene in Inhaltsverzeichnis
\setkomafont{descriptionlabel}{\bfseries\rmfamily} % für description Umgebungen
\setkomafont{captionlabel}{\small\bfseries}
\setkomafont{caption}{\small}
% Sprache: Deutsch
\usepackage[ngerman]{babel} % Silbentrennung
% PDF
\usepackage[ngerman,pdfauthor={Martin Bretschneider},  pdfauthor={Martin Bretschneider}, pdftitle={Vorlage für LaTeX}, breaklinks=true,baseurl={http://www.bretschneidernet.de/tips/thesislatex.html}]{hyperref}
\usepackage[final]{microtype} % mikrotypographische Optimierungen
\usepackage{url}
\usepackage{pdflscape} % einzelne Seiten drehen können
% Tabellen
\usepackage{multirow} % Tabellen-Zellen über mehrere Zeilen
\usepackage{multicol} % mehre Spalten auf eine Seite
\usepackage{tabularx} % Für Tabellen mit vorgegeben Größen
\usepackage{longtable} % Tabellen über mehrere Seiten
\usepackage{array}
%  Bibliographie
\usepackage{bibgerm} % Umlaute in BibTeX
% Tabellen
\usepackage{multirow} % Tabellen-Zellen über mehrere Zeilen
\usepackage{multicol} % mehre Spalten auf eine Seite
\usepackage{tabularx} % Für Tabellen mit vorgegeben Größen
\usepackage{array}
\usepackage{float}
% Bilder
\usepackage{graphicx} % Bilder
\usepackage{color} % Farben
\graphicspath{{images/}}
\DeclareGraphicsExtensions{.pdf,.png,.jpg} % bevorzuge pdf-Dateien
\usepackage{subfigure} % mehrere Abbildungen nebeneinander/übereinander
\newcommand{\subfigureautorefname}{\figurename} % um \autoref auch für subfigures benutzen
\usepackage[all]{hypcap} % Beim Klicken auf Links zum Bild und nicht zu Caption gehen
% Bildunterschrift
\setcapindent{0em} % kein Einrücken der Caption von Figures und Tabellen
\setcapwidth[c]{0.9\textwidth}
\setlength{\abovecaptionskip}{0.2cm} % Abstand der zwischen Bild- und Bildunterschrift
% Quellcode
\usepackage{listings} % für Formatierung in Quelltexten
\definecolor{grau}{gray}{0.25}
\lstset{
	extendedchars=true,
	basicstyle=\tiny\ttfamily,
	%basicstyle=\footnotesize\ttfamily,
	tabsize=2,
	keywordstyle=\textbf,
	commentstyle=\color{grau},
	stringstyle=\textit,
	numbers=left,
	numberstyle=\tiny,
	% für schönen Zeilenumbruch
	breakautoindent  = true,
	breakindent      = 2em,
	breaklines       = true,
	postbreak        = ,
	prebreak         = \raisebox{-.8ex}[0ex][0ex]{\Righttorque},
}
% linksbündige Fußboten
\deffootnote{1.5em}{1em}{\makebox[1.5em][l]{\thefootnotemark}}

\typearea{14} % typearea am Schluss berechnen lassen, damit die Einstellungen oben berücksichtigt werden
% für autoref von Gleichungen in itemize-Umgebungen
\makeatletter
\newcommand{\saved@equation}{}
\let\saved@equation\equation
\def\equation{\@hyper@itemfalse\saved@equation}
\makeatother 



% Eigene Befehle %%%%%%%%%%%%%%%%%%%%%%%%%%%%%%%%%%%%%%%%%%%%%%%%%5
% Matrix
\newcommand{\mat}[1]{
      {\textbf{#1}}
}
\newcommand{\todo}[1]{
      {\colorbox{red}{ TODO: #1 }}
}
\newcommand{\todotext}[1]{
      {\color{red} TODO: #1} \normalfont
}
\newcommand{\info}[1]{
      {\colorbox{blue}{ (INFO: #1)}}
}
% Hinweis auf Programme in Datei
\newcommand{\datei}[1]{
      {\ttfamily{#1}}
}
\newcommand{\code}[1]{
      {\ttfamily{#1}}
}
% bild mit defnierter Breite einfügen
\newcommand{\bild}[4]{
  \begin{figure}[!hbt]
    \centering
      \vspace{1ex}
      \includegraphics[width=#2]{images/#1}
      \caption[#4]{\label{img.#1} #3}
    \vspace{1ex}
  \end{figure}
}
% bild mit eigener Breite
\newcommand{\bilda}[3]{
  \begin{figure}[!hbt]
    \centering
      \vspace{1ex}
      \includegraphics{images/#1}
      \caption[#3]{\label{img.#1} #2}
      \vspace{1ex}
  \end{figure}
}


% Bild todo
\newcommand{\bildt}[2]{
  \begin{figure}[!hbt]
    \begin{center}
      \vspace{2ex}
	      \includegraphics[width=6cm]{images/todobild}
      %\caption{\label{#1} \color{red}{ TODO: #2}}
      \caption{\label{#1} \todotext{#2}}
      %{\caption{\label{#1} {\todo{#2}}}}
      \vspace{2ex}
    \end{center}
  \end{figure}
} % Importiere die Einstellungen aus der Präambel
% hier beginnt der eigentliche Inhalt
\begin{document}
\pagenumbering{Roman} % große Römische Seitenummerierung
\pagestyle{empty}

% Titelseite
\clearscrheadings\clearscrplain

\begin{center}
\begin{Huge}
Institut für Bedeutungen\\
\vspace{3mm}
\end{Huge}{\Large Universität Bunkenstedt}\\

\vspace{20mm}
\begin{Large}
Die Bedeutung der Bedeutungen unter besonderer Berücksichtigung der Bedeutungen\\
\end{Large}
\vspace{8mm}
Masterarbeit\\
\vspace{0.4cm}
\vspace{2 cm}
Max Mustermann \\
Matrikel-Nummer 12345678\\
\vspace{8cm}
\begin{tabular}{ll}
{\bf Betreuer} & Maxima Musterfrau\\
{\bf Erstprüfer}&Prof. Musterus\\
{\bf Zweitprüfer}&Prof. Mustera\\
\end{tabular}

\end{center}
\clearpage


\pagestyle{useheadings} % normale Kopf- und Fußzeilen für den Rest

\tableofcontents
\listoffigures
\listoftables

\chapter*{Symbolverzeichnis}\label{s.sym}
\addcontentsline{toc}{chapter}{Symbolverzeichnis}
\markboth{Symbolverzeichnis}{Symbolverzeichnis}
\section*{Allgemeine Symbole}\label{s.sym.alg}
\begin{flushleft}\begin{tabularx}{\textwidth}{l|X}
Symbol & Bedeutung\\\hline
$a$ & der Skalar $a$ \\
$\vec{x}$ & der Vektor $\vec{x}$\\
$\mat{A}$ & die Matrix $\mat{A}$\\
\end{tabularx}\end{flushleft}




% richtiger Inhalt
\chapter{Einleitung}
\pagenumbering{arabic} % ab jetzt die normale arabische Nummerierung

Er hörte leise Schritte hinter sich. Das bedeutete nichts Gutes. Wer würde ihm schon folgen, spät in der Nacht und dazu noch in dieser engen Gasse mitten im übel beleumundeten Hafenviertel? Gerade jetzt, wo er das Ding seines Lebens gedreht hatte und mit der Beute verschwinden wollte! Hatte einer seiner zahllosen Kollegen dieselbe Idee gehabt, ihn beobachtet und abgewartet, um ihn nun um die Früchte seiner Arbeit zu erleichtern? Oder gehörten die Schritte hinter ihm zu einem der unzähligen Gesetzeshüter dieser Stadt, und die stählerne Acht um seine Handgelenke würde gleich zuschnappen? Er konnte die Aufforderung stehen zu bleiben schon hören. Gehetzt sah er sich um. Plötzlich erblickte er den schmalen Durchgang. Blitzartig drehte er sich nach rechts und verschwand zwischen den beiden Gebäuden. Beinahe wäre er dabei über den umgestürzten Mülleimer gefallen, der mitten im Weg lag. Er versuchte, sich in der Dunkelheit seinen Weg zu ertasten und erstarrte: Anscheinend gab es keinen anderen Ausweg aus diesem kleinen Hof als den Durchgang, durch den er gekommen war. Die Schritte wurden lauter und lauter, er sah eine dunkle Gestalt um die Ecke biegen. Fieberhaft irrten seine Augen durch die nächtliche Dunkelheit und suchten einen Ausweg. War jetzt wirklich alles vorbei, waren alle Mühe und alle Vorbereitungen umsonst? Er presste sich ganz eng an die Wand hinter ihm und hoffte, der Verfolger würde ihn übersehen, als plötzlich neben ihm mit kaum wahrnehmbarem Quietschen eine Tür im nächtlichen Wind hin und her schwang. Könnte dieses der flehentlich herbeigesehnte Ausweg aus seinem Dilemma sein? Langsam bewegte er sich auf die offene Tür zu, immer dicht an die Mauer gepresst. Würde diese Tür seine Rettung werden?



\chapter{weiteres Kapitel}\label{c.weitereskapitel}
In diesem Kapitel wird einiges gemacht\footnote{wobei einiges nicht vieles heißt, ich möchte hier also keine falschen Hoffnungen wecken.} Vor allem in \autoref{s.tiefer} wird einiges gezeigt, was noch nie jemand gesehen hat. Es lohnt sich also, dranzubleiben.

\section{eine Sektion}\label{s.einesektion}
Er hörte leise Schritte hinter sich. Das bedeutete nichts Gutes. Wer würde ihm schon folgen, spät in der Nacht und dazu noch in dieser engen Gasse mitten im übel beleumundeten Hafenviertel? Gerade jetzt, wo er das Ding seines Lebens gedreht hatte und mit der Beute verschwinden wollte! Hatte einer seiner zahllosen Kollegen dieselbe Idee gehabt, ihn beobachtet und abgewartet, um ihn nun um die Früchte seiner Arbeit zu erleichtern? \todotext{das muss ich noch verfeinern, weil ich erst zur Hälfte verstanden habe} Oder gehörten die Schritte hinter ihm zu einem der unzähligen Gesetzeshüter dieser Stadt, und die stählerne Acht um seine Handgelenke würde gleich zuschnappen? Er konnte die Aufforderung stehen zu bleiben schon hören. Gehetzt sah er sich um. Plötzlich erblickte er den schmalen Durchgang. Blitzartig drehte er sich nach rechts und verschwand zwischen den beiden Gebäuden. Beinahe wäre er dabei über den umgestürzten Mülleimer gefallen, der mitten im Weg lag. Er versuchte, sich in der Dunkelheit seinen Weg zu ertasten und erstarrte\cite{weranders}: Anscheinend gab es keinen anderen Ausweg aus diesem kleinen Hof als den Durchgang, durch den er gekommen war. Die Schritte wurden lauter und lauter, er sah eine dunkle Gestalt um die Ecke biegen. Fieberhaft irrten seine Augen durch die nächtliche Dunkelheit und suchten einen Ausweg. War jetzt wirklich alles vorbei, waren alle Mühe und alle Vorbereitungen umsonst? Er presste sich ganz eng an die Wand hinter ihm und hoffte, der Verfolger würde ihn übersehen, als plötzlich neben ihm mit kaum wahrnehmbarem Quietschen eine Tür im nächtlichen Wind hin und her schwang. Könnte dieses der flehentlich herbeigesehnte Ausweg aus seinem Dilemma sein? Langsam bewegte er sich auf die offene Tür zu, immer dicht an die Mauer gepresst. Würde diese Tür seine Rettung werden?


\bild{bild}{16cm}{Test-Bild}{Test-Bild}

Die Gleichung
\begin{equation}
a^2 + b^2 = c^2
\end{equation}
ist allseits bekannt und bedarf wohl keiner weiteren Erläuterung.

Auch nicht schlecht ist \autoref{img.bild}. Aber überhaupt keinen Sinn macht \autoref{tab.sinnlos}. Hieran sieht man den Vorteil des autoref-Befehls und das so Links erstellt werden.

\begin{table}[!hbt]\vspace{1ex}\centering\begin{tabular}{|l|l|}
\hline
Formen & Städte\\
\hline
\hline
Quadrat &  Bunkenstedt \\
\hline
Dreieck &  Laggenbeck\\
\hline
Kreis &  Peine\\
\hline
Raute & Wakaluba \\
\hline
\end{tabular}
\caption{\label{tab.sinnlos}eine sinnlose Tabelle}
\vspace{2ex}\end{table}


\subsection{jetzt geht es noch tiefer}\label{s.tiefer}

Er hörte leise Schritte hinter sich. Das bedeutete nichts Gutes. Wer würde ihm schon folgen, spät in der Nacht und dazu noch in dieser engen Gasse mitten im übel beleumundeten Hafenviertel? Gerade jetzt, wo er das Ding seines Lebens gedreht hatte und mit der Beute verschwinden wollte! Hatte einer seiner zahllosen Kollegen dieselbe Idee gehabt, ihn beobachtet und abgewartet, um ihn nun um die Früchte seiner Arbeit zu erleichtern? Oder gehörten die Schritte hinter ihm zu einem der unzähligen Gesetzeshüter dieser Stadt, und die stählerne Acht um seine Handgelenke würde gleich zuschnappen? Er konnte die Aufforderung stehen zu bleiben schon hören. Gehetzt sah er sich um. Plötzlich erblickte er den schmalen Durchgang. Blitzartig drehte er sich nach rechts und verschwand zwischen den beiden Gebäuden. Beinahe wäre er dabei über den umgestürzten Mülleimer gefallen, der mitten im Weg lag. Er versuchte, sich in der Dunkelheit seinen Weg zu ertasten und erstarrte: Anscheinend gab es keinen anderen Ausweg aus diesem kleinen Hof als den Durchgang, durch den er gekommen war. Die Schritte wurden lauter und lauter, er sah eine dunkle Gestalt um die Ecke biegen. Fieberhaft irrten seine Augen durch die nächtliche Dunkelheit und suchten einen Ausweg. War jetzt wirklich alles vorbei, waren alle Mühe und alle Vorbereitungen umsonst? Er presste sich ganz eng an die Wand hinter ihm und hoffte, der Verfolger würde ihn übersehen, als plötzlich neben ihm mit kaum wahrnehmbarem Quietschen eine Tür im nächtlichen Wind hin und her schwang. Könnte dieses der flehentlich herbeigesehnte Ausweg aus seinem Dilemma sein? Langsam bewegte er sich auf die offene Tür zu, immer dicht an die Mauer gepresst. Würde diese Tür seine Rettung werden?



\begin{itemize}
\item Erstens ist das soundso,

\item dann darf man natürlich nicht vergessen und

\item das ist auch noch wichtig.
\end{itemize}


Komplexe Tabellen sind nicht sehr einfach:

\begin{table}[!hbt]\vspace{1ex}\centering
\begin{tabular}{|ll||l|l|l|l|}\hline
\multicolumn{2}{|c||}{}&\multicolumn{4}{c|}{ dies} \\
\multicolumn{2}{|c||}{}& von dort  & und dort & über hier & zu Los \\\hline\hline
\multirow{3}*{\rotatebox{90}{das}} & hier &  bla  & bla  & bla  & bla \\\cline{2-6}
& dort & bla  & bla & bla  & bla  \\\cline{2-6}
& da &  bla  & bla & bla & bla \\\hline
\end{tabular}
\caption[eine kompliziertere Tabelle]{eine kompliziertere Tabelle mit viel Beschreibungstext, der aber nicht im Tabellenverzeichnis auftauschen soll}
\vspace{2ex}\end{table}




\chapter{Zusammenfassung}\label{c.zusammenfassung}

Er hörte leise Schritte hinter sich. Das bedeutete nichts Gutes. Wer würde ihm schon folgen, spät in der Nacht und dazu noch in dieser engen Gasse mitten im übel beleumundeten Hafenviertel? Gerade jetzt, wo er das Ding seines Lebens gedreht hatte und mit der Beute verschwinden wollte! Hatte einer seiner zahllosen Kollegen dieselbe Idee gehabt, ihn beobachtet und abgewartet, um ihn nun um die Früchte seiner Arbeit zu erleichtern? Oder gehörten die Schritte hinter ihm zu einem der unzähligen Gesetzeshüter dieser Stadt, und die stählerne Acht um seine Handgelenke würde gleich zuschnappen? Er konnte die Aufforderung stehen zu bleiben schon hören. Gehetzt sah er sich um. Plötzlich erblickte er den schmalen Durchgang. Blitzartig drehte er sich nach rechts und verschwand zwischen den beiden Gebäuden. Beinahe wäre er dabei über den umgestürzten Mülleimer gefallen, der mitten im Weg lag. Er versuchte, sich in der Dunkelheit seinen Weg zu ertasten und erstarrte: Anscheinend gab es keinen anderen Ausweg aus diesem kleinen Hof als den Durchgang, durch den er gekommen war. Die Schritte wurden lauter und lauter, er sah eine dunkle Gestalt um die Ecke biegen. Fieberhaft irrten seine Augen durch die nächtliche Dunkelheit und suchten einen Ausweg. War jetzt wirklich alles vorbei, waren alle Mühe und alle Vorbereitungen umsonst? Er presste sich ganz eng an die Wand hinter ihm und hoffte, der Verfolger würde ihn übersehen, als plötzlich neben ihm mit kaum wahrnehmbarem Quietschen eine Tür im nächtlichen Wind hin und her schwang. Könnte dieses der flehentlich herbeigesehnte Ausweg aus seinem Dilemma sein? Langsam bewegte er sich auf die offene Tür zu, immer dicht an die Mauer gepresst. Würde diese Tür seine Rettung werden?


% Anhang
\begin{landscape}\begin{multicols}{2}
\appendix
\chapter{Anhang}
\section{Quelltexte}
\subsubsection*{cpu.c aus Linux 2.6.16}\label{s.cpu}\lstinputlisting[language=C]{code/cpu.c}
\end{multicols}\end{landscape}


\bibliographystyle{alphadin_martin}
\bibliography{bibliographie}


\chapter*{Erklärung}

Hiermit versichere ich, dass ich die vorliegende Arbeit selbstständig verfasst und keine anderen als die angegebenen Quellen und Hilfsmittel benutzt habe, dass alle Stellen der Arbeit, die wörtlich oder sinngemäß aus anderen Quellen übernommen wurden, als solche kenntlich gemacht und dass die Arbeit in gleicher oder ähnlicher Form noch keiner Prüfungsbehörde vorgelegt wurde.

\vspace{3cm}
Ort, Datum \hspace{5cm} Unterschrift\\

\end{document}